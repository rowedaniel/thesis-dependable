% A template for an Honours Thesis in English Language, NUS: adapted from one by Derek Lim (2016), itself adapted from a template for Carleton College comps papers by Andrew Gainer-Dewar (2013). 
% This work is licensed under the Creative Commons Attribution 4.0 International License.
% To view a copy of this licence, visit http://creativecommons.org/licenses/by/4.0/ or send a letter to Creative Commons, 444 Castro Street, Suite 900, Mountain View, California, 94041, USA.
\documentclass[twoside]{memoir}
\usepackage{nus-el-ht}

% The Latin Modern font is a modernised replacement for the classic Computer Modern. Feel free to replace this with a different font package.
\usepackage{lmodern}
% Hyperref enables click-through hyperlinks in your PDF. The hyphens option indicates where to break long URLs. 
\PassOptionsToPackage{hyphens}{url}\usepackage{hyperref}
% We use setspace to implement double-spacing, but with a specific command to leave quotes single-spaced. 
\DisemulatePackage{setspace}\usepackage{setspace}
\expandafter\def\expandafter\quote\expandafter{\quote\singlespacing}
	\doublespacing

% tikz stuff
\usepackage[usenames,dvipsnames]{color}    
\usepackage[table]{xcolor}
\usepackage{themecolors}
\usepackage{tikz} 
\usetikzlibrary{positioning, shapes.multipart, arrows, shadows, backgrounds, fit}

% code blocks
\usepackage{listings}
\lstdefinelanguage{isabelle}{%
    keywords=[1]{type_synonym,datatype,fun,abbreviation,definition,proof,lemma,theorem},
    keywordstyle=[1]\bfseries\color{isabelleouterblue},
    keywords=[2]{where,assumes,shows,and},
    keywordstyle=[2]\bfseries\color{isabelleoutergreen},
    keywords=[3]{if,then,else,case,of,SOME,let,in,O},
    keywordstyle=[3]\color{isabelleinnerblue},
    keywords=[4]{apply, by, done},
    keywordstyle=[4]\color{isabelleoutersalmon},
}
\lstset{%
  language=isabelle,
  escapeinside={&}{&},
  columns=fixed,
  extendedchars,
  basewidth={0.5em,0.45em},
  basicstyle=\ttfamily,
  mathescape,
}


% For typesetting and labelling examples, incl. glossed examples. 
\usepackage{expex} 
	\lingset{Everyex=\singlespace}

% Titling commands. 
    \title{Specification and Refinement in the Network Stack [Working Title]} % TODO: come up with better title
\author{Daniel Neshyba-Rowe}
\supervisor {Alain K{\"a}gi}
\degree{Bachelor of Arts with Honors}
\faculty{Mathematical Sciences faculty} % TODO: figure out if this is right
\dept{Computer Science} % TODO: should this be CS and Math?
% \date % TODO: set date?

\begin{document}
% First, we go into "front matter" mode.
% Among other things, this gives us Roman page numbers.
\frontmatter

% We tell LaTeX to make a title page, sections for acknowledgements, the abstract, and the table of contents (important!). 

\maketitle

\chapter{Acknowledgement}
`This page is for making acknowledgements that have a direct bearing on the HT and is not for indulging in routine gestures of politeness or sentimental attitudinising. In all things, the candidate should be guided by good taste and good sense' \pgcitep{data-refinement}{2}. % Note also the command for citing a source and a page number. 

\begin{abstract}
  % citations look like \citep{ell-ht-format}.
        Prior work led to the first proven-reliable and viable microkernel, seL4.
      We hypothesize that similar reliability is possible for a performant IoT device.
      As a proof of concept, we are creating a networked fish tank thermometer
      with a complete IPv6 network stack and formally verifying all components.

      \begin{itemize}
          \item should be written for a more general audience
          \item alternatively: provide an additional plain-language summary
      \end{itemize}
      The job of a microkernel (or operating system) is to share resources, including CPU time.
      For a long time, it was thought that the operation of a microkernel was too complex for
      any kind of formal verification in practice (concurrency makes this particularly tricky).
      However, seL4 proved this wrong by creating and verifying the seL4 microkernel.
      What we're trying to do is make a fully-fledged application that is proven from the bottom to the top.
      Our overarching hypothesis is that this is possible.
      For this thesis, we narrow the scope to just consider a piece of the networking stack (the UDP layer), and working on the verification for that piece.
      In order to do this, we must formally define how we expect the UDP layer to work,
      defining which behaviors are acceptable and which are not.
      Subsequently, we must show that our implementation in fact conforms to these expectations.


\end{abstract}

\tableofcontents

% Include the list of figures and list of tables only if you actually have figures and tables! (The * after each indicates that it should not be included in the table of contents.)
%\listoffigures*
%\listoftables*

% Next, we go into "main matter" mode.
% This resets the page numbers and uses Arabic numerals.
\mainmatter

\chapter{Introduction}


\section{Background}

This project sits within a larger body of work by seL4 (TODO: citation needed) and K{\"a}gi (citation needed).
Specifically, some exposition is needed to clarify the process of
formal verification, as well as describe the network stack.
The end goal of this project is to produce a fully functional
networked temperature sensor which records data from a fish tank,
and sends that data over the network where it will be recorded externally.
Furthermore, the goal is to formally show that all software in the
thermometer works correctly, without any bugs or flaws that could cause
it to break or report incorrect data (barring physical malfunctions of the hardware).

\subsection{Network stack}
Key to a networked temperature sensor is the code responsible for networking---communicating over the internet to other devices.
The standard protocol for network communication is defined in
a variety of RFC and IEEE specifications.
While most of the details are beyond the scope of this paper
(though not the project as a whole), the basic architecture merits some
discussion.

The Open Systems Interconnection (OSI) model describes the standard
implementation of the network stack, for use in communication between
devices.
The network stack is divided into logical layers.
Each layer has a plain-language specification
consisting of either a Request For Comment (RFC) or an IEEE document
that describes the standard interface for interacting with that layer.
At a minimum, each layer contains a \textit{wrap} and an \textit{unwrap}
endpoint, though IP and UDP in particular require more functionality.
As such, we can think of the network stack as a series of composable
encoders and decoders,
where the call to decode a packet might look like
\lstinline{udp_unwrap(ip_unwrap(eth_unwrap(hardware_packet)))}.

If everything works perfectly, then after a packet is encoded
and sent over the physical transmission medium, decoding it should
result in the same packet, unaltered.
One of the principal design features of the OSI model is that since
it is layered, this statement need only apply within a particular layer.
That is, we expect that if we call
\lstinline{y := udp_wrap(x)}, then \lstinline{udp_wrap(y)} should
result in exactly the initial packet \lstinline{x}.
% TODO: is this notation understandable?

\begin{figure}[htpb]
    \centering
    \begin{tikzpicture}[node distance=4cm]
        % styles
        \tikzstyle{cell} = [rectangle, minimum width=3cm, minimum height=1cm, text centered, draw=black]
        \tikzstyle{arrow} = [line width=2pt,->,>=latex,draw=gray]
        \tikzstyle{box} = [draw, rectangle, align=center, minimum width=11cm, inner sep=3ex]

        % wrap
        \node (app_wrap) [cell, fill=application!80] {Application on Computer A};
        \node (udp_wrap) [cell, fill=udp!80, below=30pt of app_wrap] {UDP wrap};
        \node (ip_wrap) [cell, fill=ip!80, below=30pt of udp_wrap] {IP wrap};
        \node (ethernet_wrap) [cell, fill=ethernet!80, below=30pt of ip_wrap] {Ethernet wrap};
        \node (hardware_wrap) [cell, fill=hardware!80, below=30pt of ethernet_wrap] {Hardware};

        % unwrap
        \node (app_unwrap) [cell, fill=application!80, right=10pt of app_wrap] {Application on Computer B};
        \node (udp_unwrap) [cell, fill=udp!80, below=30pt of app_unwrap] {UDP unwrap};
        \node (ip_unwrap) [cell, fill=ip!80, below=30pt of udp_unwrap] {IP unwrap};
        \node (ethernet_unwrap) [cell, fill=ethernet!80, below=30pt of ip_unwrap] {Ethernet unwrap};
        \node (hardware_unwrap) [cell, fill=hardware!80, below=30pt of ethernet_unwrap] {Hardware};

        % layers
        \node[box, label={left:\rotatebox{90}{Higher-level}}, fit = (app_wrap) (app_unwrap)] (1) {};
        \node[box, label={left:\rotatebox{90}{Transport}}, fit = (udp_wrap) (udp_unwrap)] (1) {};
        \node[box, label={left:\rotatebox{90}{Network}}, fit = (ip_wrap) (ip_unwrap)] (1) {};
        \node[box, label={left:\rotatebox{90}{Data Link}}, fit = (ethernet_wrap) (ethernet_unwrap)] (1) {};
        \node[box, label={left:\rotatebox{90}{Physical}}, fit = (hardware_wrap) (hardware_unwrap)] (1) {};

        % flow of information
        \draw [arrow] (app_wrap) -- (udp_wrap);
        \draw [arrow] (udp_wrap) -- (ip_wrap);
        \draw [arrow] (ip_wrap) -- (ethernet_wrap);
        \draw [arrow] (ethernet_wrap) -- (hardware_wrap);
        \draw [arrow] (hardware_wrap) -- (hardware_unwrap);
        \draw [arrow] (hardware_unwrap) -- (ethernet_unwrap);
        \draw [arrow] (ethernet_unwrap) -- (ip_unwrap);
        \draw [arrow] (ip_unwrap) -- (udp_unwrap);
        \draw [arrow] (udp_unwrap) -- (app_unwrap);
    \end{tikzpicture}

    
    \caption{A simplified OSI model, showing the wrap and unwrap components of each layer. Arrows show flow of data. Session, presentation, and application layers are combined into a single ``higher-level'' layer---these layers are unnecessary for our use-case of a simple thermometer.}
    \label{fig:network-stack}
\end{figure}

\subsection{seL4}
\subsection{Notation} % maybe combine with the tools section?
\begin{itemize}
    \item refinement
    \item Hoare triples
    \item big and small step semantics
    \item $\Gamma$
\end{itemize}

\section{Goals}
\begin{itemize}
    \item ultimately, verify entire network stack
    \item for now, invariant for abstract UDP saying wrap and unwrap compose to identity function.
\end{itemize}

\subsection{Simplifying assumptions}
\begin{itemize}
    % \item all layers below UDP work (prove this later)
    % \item hardware assumptions.
    %     \begin{itemize}
    %         \item no packets dropped
    %         \item no packets corrupted
    %         \item packets are received in the order they are sent
    %     \end{itemize}
    \item UDP doesn't care about correct execution of other layers---it should work correctly within its layer
    \item suppress other layers for now
    \item this means we can treat our code as running contiguously without state changes coming from external events (this will have to be dealt with more carefully when other layers are introduced and we have more PDs).
        Treat UDP wrap as running uninterrupted until completion.
        We have access to all our memory also, and will not trap or fault on memory access.
\end{itemize}

\chapter{Our Work}

\section{Tools}

\begin{itemize}
    \item Hoare logic / Hoare triples
    \item refinement book
    \item isabelle
    \item seL4 stuff (nondeterministic monads, corres, c-parser?)
\end{itemize}

\section{Proof architecture}
How do you prove your implementation correct?

\subsection{Specifications}
\begin{itemize}
    \item UDP abstract spec send
    \item UDP abstract spec recv
    \item UDP design spec send
    \item UDP design spec recv
\end{itemize}

\subsection{Refinement}
\begin{itemize}
    \item Use the refinement book's notation?
        \begin{itemize}
            \item normal variables $\vec{x}$ 
            \item representation variables $\vec{a}$ 
            \item programs $P(\mathcal{A})$ and $P(\mathcal{C})$
            \item refinement is that
                 $CI; P(\mathcal{C}); CF \subseteq AI; P(\mathcal{A}); AF$
             % \item \begin{tikzpicture}
             %         \node (AI) {$AI$ }
             %         \node (PA) {$P(\mathcal{A})$ }
             %         \node (AF) {$AF$ }
             %     \end{tikzpicture}
                 
        \end{itemize}
    \item show that design spec refines abstract spec
\end{itemize}

\subsection{Convert C to Simpl}

\chapter{A Day in the Life of}
\begin{itemize}
    \item some isabelle guidelines
\end{itemize}

\chapter{Future Work}
\begin{itemize}
    \item Make abstract and design specs for other layers
    \item c conversion for other layers
    \item c conversion for UDP?
    \item refinement for other layers
    \item seL4 proofs (specifically, PD stuff)
    \item introduce hardware problems
        \begin{itemize}
            \item think of this as relaxing assumptions about hardware
            \item e.g. packets won't be corrupted, but could arrive in weird orders
            \item note that there's probably a limit to this,
                since a packet could randomly corrupt to
                look completely legit---it's just very
                unlikely with all the error correction
                going on.
                Thus, we could model this by, e.g.,
                saying that only a limited arbitrary subset
                of the packet can be corrupted;
                or by saying any part of the packet other
                than the CRC can be corrupted.
                Pros/cons to each.
        \end{itemize}
    \item introduce linear temporal logic to have some ``always eventually''-type stuff
\end{itemize}


% \section{Trees and Glosses}
%
% Packages have documentation that explain (among other things) how to implement them. These are brief implementation examples for \texttt{expex} and \texttt{tikz-qtree}. 
%
% Glossed examples data can be presented like in (\ref{ex:missile}) below.
%
% \ex\label{ex:missile}
% \begingl
% \gla Missile cannot anyhow tzua one. //
% \glb Missile cannot carelessly fire \textsc{attr} //
% \glft `Missiles should not be carelessly fired.' //
% \endgl
% \xe
%
% A syntax tree can be presented and labelled as a figure, as with Figure \ref{fig:ticket-topic} below. 
%
% \begin{figure}[h!]
% \caption{Deriving `Ticket you got?' through topicalization}
% \centering
% \begin{tikzpicture}[scale=0.75]
% \Tree [.S$'$ \node(x){\qroof{ticket}.NP }; [.S $\emptyset$ [.VP {\qroof{you}.NP } [.V$'$ [.V got ] \node(y){t}; ] ] ] ]
% \draw[semithick, <-] (x.south)..controls +(south:5) and +(south:3)..(y.south);
% \end{tikzpicture}
% \label{fig:ticket-topic}
% \end{figure}


% If you want to include appendices, just use the \appendix command and then make chapters as normal
\appendix
\chapter{Installation notes}

See \url{https://www.overleaf.com/read/qyhckhfyfvmb} on Overleaf for useful examples of formatted text, typing in IPA, etc.

% For the bibliography style, I load `sp.bst', the style used by the LSA in the journal `Semantics and Pragmatics'. For how to include in-line citations with page numbers and other conventions, refer to nus-el-ht.sty. 
\backmatter
\bibliography{el-ht.bib}
\end{document}

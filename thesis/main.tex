% A template for an Honours Thesis in English Language, NUS: adapted from one by Derek Lim (2016), itself adapted from a template for Carleton College comps papers by Andrew Gainer-Dewar (2013). 
% This work is licensed under the Creative Commons Attribution 4.0 International License.
% To view a copy of this licence, visit http://creativecommons.org/licenses/by/4.0/ or send a letter to Creative Commons, 444 Castro Street, Suite 900, Mountain View, California, 94041, USA.
\documentclass[twoside]{memoir}
\usepackage{nus-el-ht}

% The Latin Modern font is a modernised replacement for the classic Computer Modern. Feel free to replace this with a different font package.
\usepackage{lmodern}
% Hyperref enables click-through hyperlinks in your PDF. The hyphens option indicates where to break long URLs. 
\PassOptionsToPackage{hyphens}{url}\usepackage{hyperref}
% We use setspace to implement double-spacing, but with a specific command to leave quotes single-spaced. 
\DisemulatePackage{setspace}\usepackage{setspace}
\expandafter\def\expandafter\quote\expandafter{\quote\singlespacing}
	\doublespacing

% Linguistics-specific packages.
% For typing IPA.
\usepackage{tipa} 
	\newcommand{\phonem}[1]{\textipa{/#1/}}
	\newcommand{\phonet}[1]{\textipa{[#1]}}
% For drawing syntax trees.
\usepackage{qtree}\usepackage{tikz} 
\usepackage{tikz-qtree}
	\tikzset{every tree node/.style={align=center,anchor=north}}
% For typesetting and labelling examples, incl. glossed examples. 
\usepackage{expex} 
	\lingset{Everyex=\singlespace}

% Titling commands. 
    \title{Specification and Refinement in the Network Stack [Working Title]} % TODO: come up with better title
\author{Daniel Neshyba-Rowe}
\supervisor {Alain K{\"a}gi}
\degree{Bachelor of Arts with Honors}
\faculty{Mathematical Sciences faculty} % TODO: figure out if this is right
\dept{Computer Science} % TODO: should this be CS and Math?
% \date % TODO: set date?

\begin{document}
% First, we go into "front matter" mode.
% Among other things, this gives us Roman page numbers.
\frontmatter

% We tell LaTeX to make a title page, sections for acknowledgements, the abstract, and the table of contents (important!). 

\maketitle

\chapter{Acknowledgement}
`This page is for making acknowledgements that have a direct bearing on the HT and is not for indulging in routine gestures of politeness or sentimental attitudinising. In all things, the candidate should be guided by good taste and good sense' \pgcitep{data-refinement}{2}. % Note also the command for citing a source and a page number. 

\begin{abstract}
  % citations look like \citep{ell-ht-format}.
        Prior work led to the first proven-reliable and viable microkernel, seL4.
      We hypothesize that similar reliability is possible for a performant IoT device.
      As a proof of concept, we are creating a networked fish tank thermometer
      with a complete IPv6 network stack and formally verifying all components.

      \begin{itemize}
          \item should be written for a more general audience
          \item alternatively: provide an additional plain-language summary
      \end{itemize}
      The job of a microkernel (or operating system) is to share resources, including CPU time.
      For a long time, it was thought that the operation of a microkernel was too complex for
      any kind of formal verification in practice (concurrency makes this particularly tricky).
      However, seL4 proved this wrong by creating and verifying the seL4 microkernel.
      What we're trying to do is make a fully-fledged application that is proven from the bottom to the top.
      Our overarching hypothesis is that this is possible.
      For this thesis, we narrow the scope to just consider a piece of the networking stack (the UDP layer), and working on the verification for that piece.
      In order to do this, we must formally define how we expect the UDP layer to work,
      defining which behaviors are acceptable and which are not.
      Subsequently, we must show that our implementation in fact conforms to these expectations.


\end{abstract}

\tableofcontents

% Include the list of figures and list of tables only if you actually have figures and tables! (The * after each indicates that it should not be included in the table of contents.)
%\listoffigures*
%\listoftables*

% Next, we go into "main matter" mode.
% This resets the page numbers and uses Arabic numerals.
\mainmatter

\chapter{Introduction}


\section{Background}

\subsection{seL4}

\subsection{Network stack}

\subsection{Notation} % maybe combine with the tools section?
\begin{itemize}
    \item refinement
    \item Hoare triples
    \item big and small step semantics
    \item $\Gamma$
\end{itemize}

\section{Goals}
\begin{itemize}
    \item ultimately, verify entire network stack
    \item for now, invariant for abstract UDP saying wrap and unwrap compose to identity function.
\end{itemize}

\subsection{Simplifying assumptions}
\begin{itemize}
    % \item all layers below UDP work (prove this later)
    % \item hardware assumptions.
    %     \begin{itemize}
    %         \item no packets dropped
    %         \item no packets corrupted
    %         \item packets are received in the order they are sent
    %     \end{itemize}
    \item UDP doesn't care about correct execution of other layers---it should work correctly within its layer
    \item suppress other layers for now
    \item this means we can treat our code as running contiguously without state changes coming from external events (this will have to be dealt with more carefully when other layers are introduced and we have more PDs).
        Treat UDP wrap as running uninterrupted until completion.
        We have access to all our memory also, and will not trap or fault on memory access.
\end{itemize}

\chapter{Our Work}

\section{Tools}

\begin{itemize}
    \item Hoare logic / Hoare triples
    \item refinement book
    \item isabelle
    \item seL4 stuff (nondeterministic monads, corres, c-parser?)
\end{itemize}

\section{Proof architecture}
How do you prove your implementation correct?

\subsection{Specifications}
\begin{itemize}
    \item UDP abstract spec send
    \item UDP abstract spec recv
    \item UDP design spec send
    \item UDP design spec recv
\end{itemize}

\subsection{Refinement}
\begin{itemize}
    \item Use the refinement book's notation?
        \begin{itemize}
            \item normal variables $\vec{x}$ 
            \item representation variables $\vec{a}$ 
            \item programs $P(\mathcal{A})$ and $P(\mathcal{C})$
            \item refinement is that
                 $CI; P(\mathcal{C}); CF \subseteq AI; P(\mathcal{A}); AF$
             % \item \begin{tikzpicture}
             %         \node (AI) {$AI$ }
             %         \node (PA) {$P(\mathcal{A})$ }
             %         \node (AF) {$AF$ }
             %     \end{tikzpicture}
                 
        \end{itemize}
    \item show that design spec refines abstract spec
\end{itemize}

\subsection{Convert C to Simpl}

\chapter{A Day in the Life of}
\begin{itemize}
    \item some isabelle guidelines
\end{itemize}

\chapter{Future Work}
\begin{itemize}
    \item Make abstract and design specs for other layers
    \item c conversion for other layers
    \item c conversion for UDP?
    \item refinement for other layers
    \item seL4 proofs (specifically, PD stuff)
    \item introduce hardware problems
        \begin{itemize}
            \item think of this as relaxing assumptions about hardware
            \item e.g. packets won't be corrupted, but could arrive in weird orders
            \item note that there's probably a limit to this,
                since a packet could randomly corrupt to
                look completely legit---it's just very
                unlikely with all the error correction
                going on.
                Thus, we could model this by, e.g.,
                saying that only a limited arbitrary subset
                of the packet can be corrupted;
                or by saying any part of the packet other
                than the CRC can be corrupted.
                Pros/cons to each.
        \end{itemize}
    \item introduce linear temporal logic to have some ``always eventually''-type stuff
\end{itemize}


% \section{Trees and Glosses}
%
% Packages have documentation that explain (among other things) how to implement them. These are brief implementation examples for \texttt{expex} and \texttt{tikz-qtree}. 
%
% Glossed examples data can be presented like in (\ref{ex:missile}) below.
%
% \ex\label{ex:missile}
% \begingl
% \gla Missile cannot anyhow tzua one. //
% \glb Missile cannot carelessly fire \textsc{attr} //
% \glft `Missiles should not be carelessly fired.' //
% \endgl
% \xe
%
% A syntax tree can be presented and labelled as a figure, as with Figure \ref{fig:ticket-topic} below. 
%
% \begin{figure}[h!]
% \caption{Deriving `Ticket you got?' through topicalization}
% \centering
% \begin{tikzpicture}[scale=0.75]
% \Tree [.S$'$ \node(x){\qroof{ticket}.NP }; [.S $\emptyset$ [.VP {\qroof{you}.NP } [.V$'$ [.V got ] \node(y){t}; ] ] ] ]
% \draw[semithick, <-] (x.south)..controls +(south:5) and +(south:3)..(y.south);
% \end{tikzpicture}
% \label{fig:ticket-topic}
% \end{figure}


% If you want to include appendices, just use the \appendix command and then make chapters as normal
\appendix
\chapter{Installation notes}

See \url{https://www.overleaf.com/read/qyhckhfyfvmb} on Overleaf for useful examples of formatted text, typing in IPA, etc.

% For the bibliography style, I load `sp.bst', the style used by the LSA in the journal `Semantics and Pragmatics'. For how to include in-line citations with page numbers and other conventions, refer to nus-el-ht.sty. 
\backmatter
\bibliography{el-ht.bib}
\end{document}

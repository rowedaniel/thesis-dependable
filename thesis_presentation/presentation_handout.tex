\documentclass[a4paper]{article}

\usepackage{parskip}
\usepackage[a4paper, total={7in, 8in}]{geometry}

\usepackage{color}    
\usepackage[table]{xcolor}
\usepackage{themecolors}
\usepackage{multicol}

% code blocks
\usepackage{listings}
\lstdefinelanguage{isabelle}{%
    keywords=[1]{type_synonym,datatype,fun,abbreviation,definition,proof,lemma,theorem,session,sessions,theories,text,section},
    keywordstyle=[1]\bfseries\color{isabelleouterblue},
    keywords=[2]{where,assumes,shows,and},
    keywordstyle=[2]\bfseries\color{isabelleoutergreen},
    keywords=[3]{if,then,else,case,of,SOME,let,in,O},
    keywordstyle=[3]\color{isabelleinnerblue},
    keywords=[4]{apply, by, done},
    keywordstyle=[4]\color{isabelleoutersalmon},
    morestring=[b]",
    morecomment=[n]{(*}{*)},
    % morecomment=[n]{\guilsingleft}{\guilsingright},
    stringstyle=\color{isabellequote},
    showstringspaces=false,
}
\lstdefinelanguage{file}{
}
\lstset{%
  language=isabelle,
  escapeinside={&}{&},
  columns=fixed,
  extendedchars,
  basewidth={0.5em,0.45em},
  basicstyle=\ttfamily,
  mathescape,
}

\begin{document}
\title{Isabelle reference sheet}
\maketitle

Consider a function which takes two lists of integers, and concatenates them
together.
Here's one way you could implement them, in Isabelle and in Python.

\begin{minipage}[t]{0.5\textwidth}
    \section*{Python}
    \begin{lstlisting}[language=python]
def append_lists(l1: list[int],
                 l2: list[int]
                ) -> list[int]:
    return l1 + l2 
    \end{lstlisting}

    Calling \lstinline{}\lstinline{append_lists([1], [2])}
    will return \lstinline{[1,2]}.
\end{minipage}
\begin{minipage}[t]{0.5\textwidth}
    \section*{Isabelle}
    \begin{lstlisting}[language=isabelle]
fun append_lists :: "int list $\Rightarrow$
                     int list $\Rightarrow$
                     int list" where
  "append_lists l1 l2 = l1 $@$ l2"
    \end{lstlisting}
    Calling \lstinline{}\lstinline{append_lists [1]  [2]}
    will return \lstinline{[1,2]}.
    Note that the parentheses are omitted in function calls in isabelle,
    and arguments are separated by spaces.
\end{minipage}

\vspace{40pt}
Now consider a function which compares two lists.
Here are two recursive definitions of such a function.

\begin{minipage}[t]{0.5\textwidth}
    \begin{lstlisting}[language=python]
def are_lists_same(l1: list[int],
                   l2: list[int]
                  ) -> bool:
    if l1 == [] and l2 == []:
        return True
    elif l1 == [] and l2 != []:
        return False
    elif l1 != [] and l2 == []:
        return False
    elif l1 != [] and l2 != []:
        return (
                l1[0] == l2[0] and \
                are_lists_same(l1[1:], l2[1:])
               )
    \end{lstlisting}
\end{minipage}
\begin{minipage}[t]{0.5\textwidth}
    \begin{lstlisting}[language=isabelle]
fun are_lists_same :: "int list $\Rightarrow$
                       int list $\Rightarrow$
                       bool" where
    "are_lists_same [] [] =
                            True" |
    "are_lists_same ms [] = 
                            False" |
    "are_lists_same [] ms' = 
                            False" |
    "are_lists_same (m#ms) (m'#ms') =
        (
            (m = m') $\land$ 
            are_lists_same ms ms'
        )"
    \end{lstlisting}
\end{minipage}

\end{document}

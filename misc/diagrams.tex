\documentclass[a4paper]{article}

\usepackage[utf8]{inputenc}
\usepackage{amsmath, amssymb}
\usepackage[usenames,dvipsnames]{color}    
\usepackage[table]{xcolor}
\usepackage{themecolors}
\usepackage{tikz} 
\usetikzlibrary{positioning, shapes.multipart, arrows, arrows.meta, shadows, backgrounds, fit}

\pgfdeclarelayer{background}
\pgfdeclarelayer{foreground}
\pgfsetlayers{background,main,foreground}

\begin{document}
    
    \begin{figure}
        \centering
        \scalebox{0.6}{
        \begin{tikzpicture}[node distance=4cm]
            % styles
            \tikzstyle{cell} = [rectangle, minimum width=5.4cm, minimum height=3cm, text centered, draw=black]
            \tikzstyle{arrow} = [line width=2pt,->,>=latex,draw=gray]
            \tikzstyle{box} = [draw, rectangle, align=center, minimum width=11cm, inner sep=3ex]
    
            % wrap
            \node (abstract) [cell, fill=ttorange!80] {\LARGE abstract specification};
            \node (invs) [cell, fill=ttred!80, right=140pt of abstract] {\LARGE invariants};
            \node (cspec) [cell, fill=ttyellow!80, below=90pt of abstract] {\LARGE C specification};
            % TODO: from Alain: capitalize C
            \node (ccode) [cell, fill=ttyellow!80, right=140pt of cspec] {\LARGE C code};
            % TODO: from Alain: capitalize C
            
            % \node [color=red] at (0, -150pt) {blocked access};
            % \node [color=green!200] at (-115pt, -180pt) {good access};
    
            % flow of information
            \draw [arrow] (cspec) -- (abstract) node[midway, left] {\huge \rotatebox{90}{refines}};
            \draw [arrow] (abstract) -- (invs) node[midway, above] {\huge satisfies};
            \draw [arrow] (ccode) -- (cspec) node[midway, above] {\huge translates to};
        \end{tikzpicture}}
        \caption{Updated proof architecture, including C code and a C specification.}
        \label{fig:proof-structure-c}
    \end{figure}



\begin{figure}[h]
    \centering
    \begin{tikzpicture}[node distance=4cm]
        % styles
        \tikzstyle{cell} = [rectangle, minimum width=4.6cm, minimum height=1cm, text centered, draw=black]
        \tikzstyle{arrow} = [line width=2pt,->,>=latex,draw=black]
        \tikzstyle{box} = [draw, fill=ttbggrey2, rectangle, align=center, minimum width=11cm, inner sep=3ex]

        \begin{pgfonlayer}{foreground}
        % wrap
        \node (app_wrap) [cell, fill=ttred!80] {Application on Computer A};
        \node (udp_wrap) [cell, fill=ttorange!80, below=30pt of app_wrap] {UDP wrap};
        \node (ip_wrap) [cell, fill=ttyellow!80, below=30pt of udp_wrap] {IP wrap};
        \node (ethernet_wrap) [cell, fill=ttgreen!80, below=30pt of ip_wrap] {Ethernet wrap};
        \node (hardware_wrap) [cell, fill=ttblue!80, below=30pt of ethernet_wrap] {Hardware};

        % unwrap
        \node (app_unwrap) [cell, fill=ttred!80, right=14pt of app_wrap] {Application on Computer B};
        \node (udp_unwrap) [cell, fill=ttorange!80, below=30pt of app_unwrap] {UDP unwrap};
        \node (ip_unwrap) [cell, fill=ttyellow!80, below=30pt of udp_unwrap] {IP unwrap};
        \node (ethernet_unwrap) [cell, fill=ttgreen!80, below=30pt of ip_unwrap] {Ethernet unwrap};
        \node (hardware_unwrap) [cell, fill=ttblue!80, below=30pt of ethernet_unwrap] {Hardware};
        \end{pgfonlayer}

        \begin{pgfonlayer}{background}
        % layers
        \node[box, label={left:\rotatebox{90}{Higher-level}}, fit = (app_wrap) (app_unwrap)] (1) {};
        \node[box, label={left:\rotatebox{90}{Transport}}, fit = (udp_wrap) (udp_unwrap)] (1) {};
        \node[box, label={left:\rotatebox{90}{Network}}, fit = (ip_wrap) (ip_unwrap)] (1) {};
        \node[box, label={left:\rotatebox{90}{Data Link}}, fit = (ethernet_wrap) (ethernet_unwrap)] (1) {};
        \node[box, label={left:\rotatebox{90}{Physical}}, fit = (hardware_wrap) (hardware_unwrap)] (1) {};
        \end{pgfonlayer}



        % flow of information
        \draw [arrow] (app_wrap) -- (udp_wrap);
        \draw [arrow] (udp_wrap) -- (ip_wrap);
        \draw [arrow] (ip_wrap) -- (ethernet_wrap);
        \draw [arrow] (ethernet_wrap) -- (hardware_wrap);
        \draw [arrow] (hardware_wrap) -- (hardware_unwrap);
        \draw [arrow] (hardware_unwrap) -- (ethernet_unwrap);
        \draw [arrow] (ethernet_unwrap) -- (ip_unwrap);
        \draw [arrow] (ip_unwrap) -- (udp_unwrap);
        \draw [arrow] (udp_unwrap) -- (app_unwrap);
    \end{tikzpicture}


    \caption{A simplified version of the OSI model.
    The Open Systems Interconnection (OSI) model provides a high-level description of the standard
implementation of the network stack.
Shown here are the wrap and unwrap components of each layer. Arrows show flow of data. Session, presentation, and application layers are combined into a single ``higher-level'' layer---these layers are unnecessary for our use case of a simple thermometer.}
    \label{fig:network-stack}
\end{figure}

\begin{figure}
    \centering
    \scalebox{0.65}{
    \begin{tikzpicture}[
    dot/.style = {circle, fill, minimum size=#1,
                  inner sep=0pt, outer sep=0pt},
    dot/.default = 6pt
    ]
            % styles
            \tikzstyle{arrow} = [line width=2pt,->,>=latex,draw=gray]
    
            \node (abss) [dot=100pt, fill=ttorange!80,label=\huge abstract state space] {};
            \node (abp) [dot, right=-40pt of abss] {};
            \node (abss2) [dot=100pt, fill=ttorange!80,label=\huge abstract state space, right=100pt of abss] {};
            \node (abpout) [dot=50pt, fill=ttgray!30, right=-70pt of abss2] {};
            
            \node (copout2) at (abpout) [dot=20pt, fill=ttgray2!90, right=-60pt of abss2] {};
            
            \node (coss) [dot=100pt, fill=ttyellow!80,label=below:\huge concrete state space, below=100pt of abss] {};
            \node (cop) [dot, right=-40pt of coss] {};
            \node (coss2) [dot=100pt, fill=ttyellow!80,label=below:\huge concrete state space, right=100pt of coss] {};
            \node (copout) [dot=20pt, fill=ttgray!90, right=-60pt of coss2] {};
    
            \draw [arrow] (abp) -- (abpout) node[midway, above] {\large abstract program};
            \draw [arrow] (cop) -- (copout) node[midway, above] {\large concrete program};
            \draw [arrow, <->] (copout) -- (copout2) node[midway, right] {\large \rotatebox{90}{equivalent}};
            \draw [arrow, <->] (abp) -- (cop) node[midway, right] {\large \rotatebox{90}{equivalent}};
    \end{tikzpicture}}
\end{figure}

\end{document}
